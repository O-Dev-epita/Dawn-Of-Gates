\newglossaryentry{tag}
{
  name=tag,
  description={Propriété d'un objet Unity, permettant de donner un nom à un objet, afin de pouvoir le retrouver plus facilement}
}

\newglossaryentry{CSS}
{
  name=CSS,
  description={Cascading Style Sheets (Feuilles de style en cascades). Language permettant de mettre en forme un site web}
}

\newglossaryentry{HTML}
{
  name=HTML,
  description={HyperText Markup Language. Le HTML est un langage de description permettant de réaliser le contenu d'un site web}
}

\newglossaryentry{Ruby}
{
  name=Ruby,
  description={Langage de programmation orienté objet interprété avec une syntaxe légère}
}

\newglossaryentry{OpenGL}
{
  name=OpenGL,
  description={API multiplateformes pour la réalisation de graphismes en 3D}
}

\newglossaryentry{C Sharp}
{
  name=C Sharp,
  description={Language orienté objet, haut niveau, pré compilé.}
}

\newglossaryentry{Javascript}
{
  name=Javascript,
  description={Langage de script interprété par le navigateur permettant d'animer une page web}
}

\newglossaryentry{PHP}
{
  name=PHP,
  description={Langage de script interprété par le serveur, il permet de réaliser des sites web dynamiques}
}

\newglossaryentry{SQL}
{
  name=SQL,
  description={Langage permettant de communiquer avec une base de donnée}
}

\newglossaryentry{C}
{
  name=C,
  description={Langage de programmation compilé, procédural, très bas niveau et très rapide}
}

\newglossaryentry{C++}
{
  name=C++,
  description={Langage de programmation orienté objet, il a toutes les fonctionnalités du C, avec en plus les classes, les templates, et bien d'autres}
}

\newglossaryentry{framerates}
{
  name=framerates,
  description={Fréquence de rafraichissement d'un écran}
}

\newglossaryentry{Caml}
{
  name=Caml,
  description={Langage de programmation automatiquement typé, pouvant être interprété ou compilé}
}

\newglossaryentry{assembleur}
{
  name=assembleur,
  description={Language constitué d'une succession d'instruction qui sont converties en language machine pour être exécutées par le proccesseur. Il existe un language assembleur différent pour chaque type de proccesseur}
}

\newglossaryentry{Java}
{
  name=Java,
  description={Langage de programmation précompilé et haut niveau. Il s'exécute sur une machine virtuelle, indiféremment de l'OS utilité}
}

\newglossaryentry{Python}
{
  name=Python,
  description={Langage de programmation interprété, ayant une syntaxe très légère, permettant de faire beaucoup en peu de lignes}
}

\newglossaryentry{template}
{
  name=template,
  description={Modèle}
}

\newglossaryentry{shuriken}
{
  name=shuriken,
  description={Arme de jet asiatique}
}

\newglossaryentry{SDK}
{
  name=SDK,
  description={Sofware Development Kit (Kit de développemnt logicel) : C'est un logiciel d'aide au développement}
}

\newglossaryentry{FPS}
{
  name=FPS,
  description={First Person Shooter (Jeu de tir à la première personne)}
}

\newglossaryentry{raycast}
{
  name=raycast,
  description={Fonctionnalité d'Unity utilisé dans les scripts. Elle permet de savoir quels objets sont sur la trajectoire d'un vecteur défini}
}

\newglossaryentry{shader}
{
  name=shader,
  description={Script executé par la carte graphique permettanr de donner des effets visuels particuliers}
}

\newglossaryentry{texture renderer}
{
  name=texture renderer,
  description={Objet d'unity. Contrairement à une texture normale, celle-ci peut changer durer l'exécution}
}

\newglossaryentry{box collider}
{
  name=box collider,
  description={Objet 3D de la forme d'un parallélépipède rectangle permettant de gérer les collisions avec d'autres objets.}
}

\newglossaryentry{slow motion}
{
  name=slow motion,
  description={Ralentissement du temps}
}

\newglossaryentry{GUI}
{
  name=GUI,
  description={Graphical User Interface (Interface graphique d'utilisateur)}
}

\newglossaryentry{HTTP}
{
  name=HTTP,
  description={Hypertext Transfer Protocol : protocole de transfert permettant au navigateur de communiquer avec un site web}
}

\newglossaryentry{animator}
{
  name=animator,
  description={Menu unity permettant de gérer les animations d'un objet, ainsi que les différentes transitions}
}

\newglossaryentry{GameObject}
{
  name=GameObject,
  description={Dans Unity, un GameObject représente n'importe quelle type d'objet inclus dans le jeu, un personnage, une arme, une lumière, ou autre}
}

\newglossaryentry{framework}
{
  name=framework,
  description={Ensemble de bibliothéques couvrant généralement en grande partie un ou plusieurs domaines, ayant un niveau d'absraction variant de faible à élevé, afin qu'un développeur puisse l'utiliser ou le compléter facilement }
}

\newglossaryentry{WYSIWYG}
{
  name=WYSIWYG,
  description={What You See Is What You Get (Ce que vous voyez est ce que vous obtenez). Logiciel de production permettant d'arriver facilement au résultat souhaité.}
}

\newglossaryentry{main}
{
  name=main,
  description={Nom de la fonction de début de programme. De nombreux languages de programmation possèdent cette fonction, qui marque le point de départ du programme}
}

\newglossaryentry{serialize}
{
  name=serialize,
  description={En programmation, permet de transformer une variable de n'importe quelle type en données binaires}
}

\newglossaryentry{unserialize}
{
  name=unserialize,
  description={En programmation, permet de récuper une variable à partir de sa version sérializée}
}

\newglossaryentry{timestamp}
{
  name=timestamp,
  description={Le nombre de secondes écoles depuis le premier janvier 1970}
}

\newglossaryentry{Cocoa}
{
  name=Cocoa,
  description={Framework utilisable sur l'IDE XCode en Objective-C, couvrant la réalisation d'interfaces pour Os X et iOs}
}


\newglossaryentry{quaternions}
{
  name=quaternions,
  description={En mathématiques ce sont des nombres dits hyper complexes. Ils sont utilisés pour représenter la rotation en 3D}
}

\newglossaryentry{Qt}
{
  name=Qt,
  description={Framework permettant la réalisation d'interphaces graphiques multi-plateformes en C++}
}

\newglossaryentry{url}
{
  name=url,
  description={Uniform Ressource Locator : texte permettant d'identifier une ressource sur internet}
}