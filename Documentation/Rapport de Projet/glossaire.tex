\newglossaryentry{tag}
{
  name=tag,
  description={Propriété d'un objet Unity, permettant de donner un nom à un objet, afin de pouvoir le retrouver plus facilement}
}

\newglossaryentry{css}
{
  name=css,
  description={Cascading Style Sheets (Feuilles de style en cascades). Language permettant de mettre en forme un site web}
}

\newglossaryentry{template}
{
  name=template,
  description={Modèle}
}

\newglossaryentry{shuriken}
{
  name=shuriken,
  description={Arme de jet asiatique}
}

\newglossaryentry{SDK}
{
  name=SDK,
  description={Sofware Development Kit (Kit de développemnt logicel) : C'est un logiciel d'aide au développement}
}

\newglossaryentry{FPS}
{
  name=FPS,
  description={First Person Shooter (Jeu de tir à la première personne)}
}

\newglossaryentry{raycast}
{
  name=raycast,
  description={Permet de savoir quels objets sont sur la trajectoire de ce vecteur}
}

\newglossaryentry{shader}
{
  name=shader,
  description={Script executé par la carte graphique permettanr de donner des effets visuels particuliers}
}

\newglossaryentry{texture renderer}
{
  name=texture renderer,
  description={Objet d'unity. Contrairement à une texture normale, celle-ci peut changer durer l'exécution}
}

\newglossaryentry{box collider}
{
  name=box collider,
  description={Objet 3D de la forme d'un parallélépipède rectangle permettant de gérer les collisions avec d'autres objets.}
}

\newglossaryentry{slow motion}
{
  name=slow motion,
  description={Ralentissement du temps}
}

\newglossaryentry{GUI}
{
  name=GUI,
  description={Graphical User Interface (Interface graphique d'utilisateur)}
}

\newglossaryentry{HTTP}
{
  name=HTTP,
  description={Hypertext Transfer Protocol : protocole de transfert permettant au navigateur de communiquer avec un site web}
}

\newglossaryentry{animator}
{
  name=animator,
  description={Menu unity permettant de gérer les animations d'un objet, ainsi que les différentes transitions}
}